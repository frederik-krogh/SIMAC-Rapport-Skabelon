%------------------------------------------------
%   INDHOLD: Guide til LaTeX og SIMAC-skabelonen
%   Forord, introduktion, eksempler og kilder
%------------------------------------------------

\section*{Forord}

Denne skabelon er udviklet af \textbf{Frederik Krogh} som et personligt værktøj til at skrive rapporter på maskinmesteruddannelsen.
Jeg deler den her, så andre kan bruge den, hvis de har lyst. Skabelonen er ikke officielt tilknyttet SIMAC, og jeg har desværre ikke mulighed for at hjælpe med hverken LaTeX eller opgaveskrivning.

Hvis du synes, den er nyttig, er du meget velkommen til at bruge den som grundlag for dine egne rapporter og tilpasse den, som du ønsker.

\newpage

%------------------------------------------------
\section{Introduktion til LaTeX}

\subsection{Hvad er LaTeX?}
LaTeX er et tekstbehandlingssystem udviklet til at skrive tekniske og videnskabelige dokumenter.
I stedet for at redigere tekst visuelt som i Word, skriver du tekst og kommandoer i kode, som derefter omdannes til en PDF.

LaTeX bruges verden over på tekniske uddannelser og i industrien, fordi det sikrer ensartet formatering og professionelle rapporter – perfekt til maskinmesterstuderende.

\subsection{Hvorfor bruge LaTeX?}
\begin{itemize}
  \item \textbf{Ensartet formatering:} Overskrifter, figurer og tabeller nummereres automatisk.
  \item \textbf{Automatiske kilder:} BibTeX håndterer litteraturlisten for dig.
  \item \textbf{Fuld kontrol:} Du kan præcist styre, hvordan dit dokument ser ud.
  \item \textbf{Perfekt til teknik:} Ideelt til formler, måleresultater, diagrammer og tekniske beregninger.
\end{itemize}

\subsection{Hvor skriver man LaTeX?}
Hvis du aldrig har programmeret før, så brug editoren \textbf{Crixit} (\href{https://crixit.dk}{crixit.dk}).
Den fungerer direkte i browseren, kræver ingen installation og er danskudviklet.

Du kan bruge \textbf{ChatGPT} som sparringspartner til LaTeX.
For eksempel kan du spørge:
\textit{"Hvordan laver jeg en tabel i LaTeX med tre kolonner?"}

\subsection{Onlineværktøjer til LaTeX}
\begin{itemize}
  \item \href{https://www.tablesgenerator.com/latex_tables}{Tables Generator} – lav tabeller uden at skrive kode.
  \item \href{https://latexdraw.com}{LaTeXDraw} – lav tekniske figurer.
  \item \href{https://detexify.kirelabs.org/classify.html}{Detexify} – tegn et symbol og få LaTeX-koden.
\end{itemize}

\newpage

%------------------------------------------------
\section{Kom godt i gang med SIMAC-skabelonen}

\subsection{Download og opsætning}
Hvis du aldrig har brugt GitHub før, kan du hente skabelonen som en ZIP-fil:

\begin{enumerate}
  \item Gå til \href{https://github.com/frederik-krogh/SIMAC-Rapport-Skabelon}{https://github.com/frederik-krogh/SIMAC-Rapport-Skabelon}.
  \item Klik på den grønne knap \textbf{“Code”} og vælg \textbf{“Download ZIP”}.
  \item Pak ZIP-filen ud på din computer.
  \item Åbn mappen i Crixit.
  \item Redigér i \texttt{report.tex}.
\end{enumerate}

\subsection{Skabelonens struktur}
\begin{itemize}
  \item \texttt{main.tex} – hovedfilen, som samler hele dokumentet. Det er her du retter opgavens navn, gruppemedlemmer osv.
  \item \texttt{report.tex} – selve indholdet af rapporten, altså det du læser her.
  \item \texttt{figurer/} – her placeres billeder og diagrammer.
  \item \texttt{kilder.bib} – din litteraturliste.
\end{itemize}

De øvrige filer og mapper skal du i udgangspunktet ikke redigere i.

LaTeX samler selv det hele og skaber et ensartet og professionelt dokument.

\newpage

%------------------------------------------------
\section{Eksempler på brug af skabelonen}

%----------------------------------------------------------------------------------------
%  SECTIONS
%----------------------------------------------------------------------------------------

\section{Section Title} % Top level section

Lorem ipsum dolor sit amet, consectetur adipiscing elit. Aliquam auctor mi risus, quis tempor libero hendrerit at. Duis hendrerit placerat quam et semper. Nam ultricies metus vehicula arcu viverra, vel ullamcorper justo elementum. Pellentesque vel mi ac lectus cursus posuere et nec ex. Fusce quis mauris egestas lacus commodo venenatis. Ut at arcu lectus. Donec et urna nunc. Morbi eu nisl cursus sapien eleifend tincidunt quis quis est. Donec ut orci ex. Praesent ligula enim, ullamcorper non lorem a, ultrices volutpat dolor. Nullam at imperdiet urna. Pellentesque nec velit eget est pretium.\sidenote{This is a sidenote. This template features a large margin specifically so you can put notes, figures, tables and other things into it as additional material to the main content in the text block.}

Donec in elit ac ante vestibulum rhoncus. Pellentesque ligula tortor, aliquet malesuada nulla tristique vitae. Aliquam mi sem, varius eu pellentesque et, tristique nec quam. Vestibulum pellentesque in dui et venenatis. Sed malesuada elit pellentesque sapien aliquet porta. In at facilisis diam. Duis id ante tellus.\sidenote[][2cm]{This sidenote has been pushed down the page manually with an optional parameter, otherwise it would be right under the one above.} % This first optional argument to a sidenote is the symbol to use (leave this empty for automatic numbering) and the second is the vertical offset (positive is down, negative is up)

\subsection{Subsection Title} % Second level section

In diam libero, vulputate quis accumsan non, auctor in ipsum. Praesent cursus velit eget lacus sodales porta. Proin quis risus ut velit euismod scelerisque ut sed neque. Cras sagittis, dolor ac ullamcorper auctor, tortor dui facilisis diam, at sagittis nisi ipsum a neque. Nullam vel mattis nisi. Ut interdum ut diam at ornare. Nulla ultrices elit justo, vitae tristique massa vulputate sit amet.

\nonumsidenote{This sidenote isn't numbered in the text or margin. This is useful for notes that apply anywhere on the page instead of one particular place.}Vestibulum erat felis, cursus vitae convallis ac, commodo eu nisi. Nulla facilisi. Mauris dignissim nisi felis, a mollis ex accumsan vel. Suspendisse bibendum vitae nibh in suscipit. Vestibulum et finibus eros. Nulla facilisi. Cras luctus aliquam finibus. In nec justo nec orci malesuada faucibus.

\subsubsection{Subsubsection Title} % Third level section

\begin{fullwidth} % Use the whole page width
  \textit{This is an example of a full width paragraph\ldots} Curabitur id placerat orci. Vivamus pulvinar augue ac feugiat blandit. Donec in ultricies mi. Nam eu lacus ac augue aliquet consectetur. Praesent dui risus, sollicitudin nec felis ut, posuere ultricies dolor. Sed massa nulla, dignissim eget sem sit amet, eleifend fermentum dui. Phasellus consequat sem vel turpis finibus, a aliquam risus malesuada.
\end{fullwidth}

Maecenas consectetur metus at tellus finibus condimentum. Proin arcu lectus, ultrices non tincidunt et, tincidunt ut quam. Integer luctus posuere est, non maximus ante dignissim quis. Nunc a cursus erat. Curabitur suscipit nibh in tincidunt sagittis. Nam malesuada vestibulum quam id gravida. Proin ut dapibus velit. Vestibulum eget quam quis ipsum semper convallis. Duis consectetur nibh ac diam dignissim, id condimentum enim dictum. Nam aliquet ligula eu magna pellentesque, nec sagittis leo lobortis. Aenean tincidunt dignissim egestas. Morbi efficitur risus ante, id tincidunt odio pulvinar vitae.

\paragraph{Paragraph Title} % Fourth level section

Lorem ipsum dolor sit amet, consectetur adipiscing elit. Aliquam auctor mi risus, quis tempor libero hendrerit at. Duis hendrerit placerat quam et semper. Nam ultricies metus vehicula arcu viverra, vel ullamcorper justo elementum. Pellentesque vel mi ac lectus cursus posuere et nec ex.

The section titles below show how multi-line section titles look at the 3 top levels.

\section[Short version of long section title]{Fusce eleifend porttitor arcu, id accumsan elit pharetra eget} % Use the optional parameter to the \section command to specify a shorter version of the title for the table of contents

Lorem ipsum dolor sit amet, consectetur adipiscing elit. \nonumsidenote[-2cm]{Section, subsection and subsubsection titles can span multiple lines, as shown here. Make sure to put a shorter version of these long titles in the optional parameter to the section commands so the title output to the table of contents is the short version.}

\subsection[Short version of long subsection title]{Phasellus sit amet enim efficitur, aliquam nulla id, lacinia mauris viverra libero ac magna}

Lorem ipsum dolor sit amet, consectetur adipiscing elit.

\subsubsection{In mi mauris, finibus non faucibus non, imperdiet nec leo. In erat arcu, tincidunt nec aliquam et, volutpat eget}

Lorem ipsum dolor sit amet, consectetur adipiscing elit.

%----------------------------------------------------------------------------------------
%  FONTS
%----------------------------------------------------------------------------------------

\section{Font Examples}

\subsection{Font Sizes}

{\tiny \textbackslash tiny} {\scriptsize \textbackslash scriptsize} {\footnotesize \textbackslash footnotesize} {\small \textbackslash small}\\
{\normalsize \textbackslash normalsize}\nonumsidenote[-1cm]{The default font size for the document is 12pt, represented by \textbackslash normalsize. The standard LaTeX font size commands modify this to be smaller or larger as needed.}\\
{\large \textbackslash large} {\Large \textbackslash Large} {\LARGE \textbackslash LARGE} {\huge \textbackslash huge} {\Huge \textbackslash Huge}

\subsection{Font Families}

\textsf{IBM Plex Sans Text}\nonumsidenote{The sans family is the default, as is standard in the business world. Use the serif family to accentuate text, such as for quotations. The mono family is best used where it's important that all characters are the same width, such as for numbers in a table or for code.}

\textrm{IBM Plex Serif Text}

\texttt{IBM Plex Mono Text}

\subsection{Font Weights}

\textel{ExtraLight} \textl{Light} Normal \textsb{SemiBold} \textbf{Bold}

\subsection{Condensed Fonts}

Plex Sans Normal\nonumsidenote{Condensed fonts can be useful if horizontal space is at a premium. You might want to use the condensed font in a wide table.}

{\plexsanscondensed Plex Sans Condensed}

%----------------------------------------------------------------------------------------
%  QUOTATIONS
%----------------------------------------------------------------------------------------

\section{Quotations}

Proin mollis urna posuere fringilla. Curabitur finibus, neque vitae vestibulum vestibulum, dolor sapien tincidunt augue, vel porta mauris metus nec mauris. Integer erat magna, porta id erat sed, lacinia volutpat erat. Nulla fermentum tellus arcu, eu iaculis ipsum malesuada aliquam. Duis et lacus maximus, consectetur metus et, eleifend arcu. Vestibulum condimentum diam vitae diam tincidunt viverra.\sidenotequote[-1cm]{\textbf{\LARGE ``}Lorem ipsum dolor sit amet, consectetur adipiscing elit. Praesent porttitor arcu luctus, imperdiet urna iaculis, mattis eros. Pellentesque iaculis odio vel nisl ullamcorper, nec faucibus ipsum molestie.\textbf{''}\\[4pt]\hfill--- John Smith, 1972} % Example margin quotation using the \sidenotequote custom command

\begin{quote}
  \textbf{\LARGE ``}Lorem ipsum dolor sit amet, consectetur adipiscing elit. Praesent porttitor arcu luctus, imperdiet urna iaculis, mattis eros. Pellentesque iaculis odio vel nisl ullamcorper, nec faucibus ipsum molestie. Sed dictum nisl non aliquet porttitor. Etiam vulputate arcu dignissim, finibus sem et, viverra nisl. Aenean luctus congue massa, ut laoreet metus ornare in.\textbf{''}

  \hfill--- John Smith, 1972
\end{quote}

Suspendisse tempus odio sit amet volutpat suscipit. Pellentesque ornare libero lacus, non fringilla dolor placerat in. Ut maximus ullamcorper lectus, a pharetra mi sagittis aliquet. Scelerisque augue sed mi fringilla, vel dapibus ligula finibus. Sed ornare velit sem, ac venenatis velit dignissim. Vestibulum ultrices mi at tincidunt condimentum.

%----------------------------------------------------------------------------------------
%  TABLES
%----------------------------------------------------------------------------------------

\section{Table Examples}

This statement automatically references the table below using its label: Table \ref{tab:example}.

%------------------------------------------------

\begin{margintable} % Use the margintable environment for tables to be output to the margin
  \footnotesize % Reduce the font size in the table as space is at a premium
  \caption{Margin table caption.}
  \begin{tabular}{L{0.22\linewidth} C{0.22\linewidth} R{0.25\linewidth}}
    \toprule
    \textbf{Year} & \textbf{Qtr.} & \textbf{Perf.}\\
    \midrule
    20XX & Q1 & 0.5\%\\
    20XX & Q2 & 26.5\%\\
    20XX & Q1 & 35.4\%\\
    20XX & Q4 & 41.3\%\\
    \bottomrule
  \end{tabular}
\end{margintable}

%------------------------------------------------

\begin{table}[H] % [H] forces the table to be output where it is defined in the code (it suppresses floating)
  \caption{Text block table caption.}
  \begin{tabular}{L{0.35\linewidth} L{0.38\linewidth} L{0.16\linewidth}}
    \toprule
    \textbf{Prospect} & \textbf{Industry} & \textbf{Revenue} \\
    \midrule
    Gerlach Inc & Business Development & \$3M\\
    Doyle and Sons & Law & \$1M\\
    Heathcote Group & Consulting & \$12M\\
    Goyette Inc & Advertising & \$5M\\
    Holzdeppe GmbH & Manufacturing & \$23M\\
    Bienias AG & Accounting & \$2.5M\\
    \bottomrule
  \end{tabular}
  \label{tab:example}
\end{table}

%------------------------------------------------

\begin{table*} % Use the table* environment for full width tables
  \caption{Full width table caption.}
  \begin{tabular}{C{0.03\linewidth} L{0.25\linewidth} L{0.27\linewidth} L{0.16\linewidth} L{0.16\linewidth}}
    \toprule
    \textit{\#} & \textbf{Prospect} & \textbf{Industry} & \textbf{Revenue} & \textbf{Employees} \\
    \midrule
    \textit{1} & Gerlach Inc & Business Development & \$3M & 65\\
    \textit{2} & Doyle and Sons & Law & \$1M & 15\\
    \textit{3} & Heathcote Group & Consulting & \$12M & 250\\
    \textit{4} & Goyette Inc & Advertising & \$5M & 100\\
    \textit{5} & Holzdeppe GmbH & Manufacturing & \$23M & 75\\
    \textit{6} & Bienias AG & Accounting & \$2.5M & 40\\
    \bottomrule
  \end{tabular}
\end{table*}

%----------------------------------------------------------------------------------------
%  FIGURES
%----------------------------------------------------------------------------------------

\section{Figure Examples}

This statement automatically references the figure below using its label: Figure \ref{fig:example}.

%------------------------------------------------

\begin{marginfigure} % Use the marginfigure environment for figures to be output to the margin
  \includegraphics[width=\linewidth]{placeholder.jpg}
  \caption{Margin figure caption.}
\end{marginfigure}

%------------------------------------------------

\begin{figure}[H] % [H] forces the figure to be output where it is defined in the code (it suppresses floating)
  \includegraphics[width=\linewidth]{ARR.pdf}
  \caption{Text block figure caption.}
  \label{fig:example} % Label for referencing this figure in the text automatically
\end{figure}

%------------------------------------------------

\begin{figure*} % Use the figure* environment for full width figures
  \includegraphics[width=\linewidth]{ARR.pdf}
  \caption{Full width figure caption.}
\end{figure*}

%----------------------------------------------------------------------------------------
%  LISTS
%----------------------------------------------------------------------------------------

\section{List Examples}

\subsection{Bullet Point List}

\nonumsidenote{Bullet point lists can also be created in the margin. For these, we can remove the usual left margin to increase the available horizontal space:\\\medskip
  \begin{itemize}[leftmargin=*]
    \item Bullet item one.
    \item Bullet item two.
    \item Bullet item three.
  \end{itemize}}Lorem ipsum dolor sit amet, consectetur adipiscing elit. Praesent porttitor arcu luctus, imperdiet urna iaculis, mattis eros. Pellentesque iaculis odio vel nisl ullamcorper, nec faucibus ipsum molestie. Sed dictum nisl non aliquet porttitor.

\begin{itemize}
  \item First bullet point item
    \begin{itemize}
      \item First indented bullet point item
      \item Second indented bullet point item
        \begin{itemize}
          \item First second-level indented bullet point item
          \item Second second-level indented bullet point item
        \end{itemize}
      \item Third indented bullet point item
    \end{itemize}
  \item Second bullet point item
  \item Third bullet point item
\end{itemize}

Etiam vulputate arcu dignissim, finibus sem et, viverra nisl. Aenean luctus congue massa, ut laoreet metus ornare in. Nunc fermentum nisi imperdiet lectus tincidunt vestibulum at ac elit. Nulla mattis nisl eu malesuada suscipit.

%------------------------------------------------

\subsection{Numbered List}

\nonumsidenote[-2.5cm]{Numbered lists can also be created in the margin. For these, we can remove the usual left margin to increase the available horizontal space:\\\medskip
  \begin{enumerate}[leftmargin=*]
    \item Numbered item one.
    \item Numbered item two.
    \item Numbered item three.
  \end{enumerate}}Lorem ipsum dolor sit amet, consectetur adipiscing elit. Praesent porttitor arcu luctus, imperdiet urna iaculis, mattis eros. Pellentesque iaculis odio vel nisl ullamcorper, nec faucibus ipsum molestie. Sed dictum nisl non aliquet porttitor.

\begin{enumerate}
  \item First numbered item
    \begin{enumerate}
      \item First indented numbered item
      \item Second indented numbered item
        \begin{enumerate}
          \item First second-level indented numbered item
          \item Second second-level indented numbered item
        \end{enumerate}
      \item Third indented numbered item
    \end{enumerate}
  \item Second numbered item
  \item Third numbered item
\end{enumerate}

Etiam vulputate arcu dignissim, finibus sem et, viverra nisl. Aenean luctus congue massa, ut laoreet metus ornare in. Nunc fermentum nisi imperdiet lectus tincidunt vestibulum at ac elit. Nulla mattis nisl eu malesuada suscipit.

%------------------------------------------------

\subsection{Description List}

\nonumsidenote{Description lists can also be created in the margin:\\\medskip
  \begin{description}
    \item[A1] Description item one.
    \item[B1] Description item two.
    \item[C1] Description item three.
  \end{description}}Lorem ipsum dolor sit amet, consectetur adipiscing elit. Praesent porttitor arcu luctus, imperdiet urna iaculis, mattis eros. Pellentesque iaculis odio vel nisl ullamcorper, nec faucibus ipsum molestie. Sed dictum nisl non aliquet porttitor.

\begin{description}
  \item[Item One] Lorem ipsum dolor sit amet, consectetur adipiscing elit. Praesent porttitor arcu luctus, imperdiet urna iaculis, mattis eros. Pellentesque iaculis odio vel nisl ullamcorper, nec faucibus ipsum molestie.
  \item[Item Two] Sed dictum nisl non aliquet porttitor.
    \begin{description}
      \item[Subitem] Maecenas consectetur metus at tellus finibus condimentum. Proin arcu lectus, ultrices non tincidunt et, tincidunt ut quam. Integer luctus posuere est, non maximus ante dignissim quis.
        \begin{description}
          \item[Subsubitem] Maecenas consectetur metus at tellus finibus condimentum. Proin arcu lectus, ultrices non tincidunt et, tincidunt ut quam. Integer luctus posuere est, non maximus ante dignissim quis.
        \end{description}
    \end{description}
  \item[Item Three] Etiam vulputate arcu dignissim, finibus sem et, viverra nisl. Aenean luctus congue massa, ut laoreet metus ornare in. Nunc fermentum nisi imperdiet lectus tincidunt vestibulum at ac elit. Nulla mattis nisl eu malesuada suscipit.
\end{description}

Etiam vulputate arcu dignissim, finibus sem et, viverra nisl. Aenean luctus congue massa, ut laoreet metus ornare in.

%----------------------------------------------------------------------------------------
%  REFERENCING CITATIONS
%----------------------------------------------------------------------------------------

\section{Referencing Citations}

This statement requires citation \autocite{githubskabelon}.

This statement requires multiple citations \autocite{simacbog2021, crixit2024}.

This short citation is in the margin\sidecite{tablesgenerator}.

This long citation is in the margin\fullsidecite{energistyrelsen2023}.

This statement has an in-text citation: \textcite{tablesgenerator}.

%----------------------------------------------------------------------------------------
%  LINKS
%----------------------------------------------------------------------------------------

\section{Link Examples}

This is a URL link: \href{https://www.duckduckgo.com}{DuckDuckGo}.\nonumsidenote{Links can be clicked in the PDF to navigate to the linked website or email address.}

This is a email link: \href{mailto:example@example.com}{example@example.com}.

This is a monospaced URL link: \url{https://duckduckgo.com}.

%----------------------------------------------------------------------------------------
%  EQUATIONS
%----------------------------------------------------------------------------------------

\section{Equation}

\begin{equation}
  \cos^3 \theta =\frac{1}{4}\cos\theta+\frac{3}{4}\cos 3\theta
  \label{eq:example}
\end{equation}

This statement automatically references the equation above using its label: Equation \ref{eq:example}.

%----------------------------------------------------------------------------------------
%  INTERNATIONAL SUPPORT
%----------------------------------------------------------------------------------------

\section{International Support}

àáâäãåèéêëìíîïòóôöõøùúûüÿýñçčšž\nonumsidenote{Plex is a very high quality typeface produced by IBM. It includes extensive international support and characters.}

ÀÁÂÄÃÅÈÉÊËÌÍÎÏÒÓÔÖÕØÙÚÛÜŸÝÑ

ßÇŒÆČŠŽ

%----------------------------------------------------------------------------------------
%  CODE
%----------------------------------------------------------------------------------------

\section{Displaying Code}

The block below is a code listing. It displays code in an easy to use way with line numbers for quick reference to specific parts of the code.

\begin{lstlisting}
{
  "city": [
    {
      "id": 1,
      "name": "Toronto",
      "country": "Canada",
      "population": 6200000
    },
    {
      "id": 2,
      "name": "New York",
      "country": "United States of America",
      "population": 8800000
    }
  ]
}
\end{lstlisting}

\newpage

%------------------------------------------------
\section{Tips og tricks}

\begin{itemize}
  \item Gem figurer i formatet \texttt{.pdf} eller \texttt{.png}.
  \item Brug korte filnavne uden mellemrum.
  \item Kompiler to gange for at opdatere kilder og figurer.
  \item Brug \texttt{\textbackslash newpage} for at starte på en ny side.
  \item Angiv altid kilder for data, figurer og tekst.
\end{itemize}

\newpage

%------------------------------------------------
\section{Konklusion}

LaTeX kan virke teknisk i starten, men når man først forstår strukturen, er det et effektivt og professionelt værktøj.
Denne skabelon gør det let at komme i gang og sikrer, at rapporten får et ensartet og flot udtryk.

Brug skabelonen som et værktøj til at fokusere på det faglige indhold – resten tager LaTeX sig af.
